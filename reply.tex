% Created 2021-05-01 Sat 15:49
% Intended LaTeX compiler: pdflatex
\documentclass[11pt]{article}
\usepackage[latin1]{inputenc}
\usepackage[T1]{fontenc}
\usepackage{graphicx}
\usepackage{grffile}
\usepackage{longtable}
\usepackage{wrapfig}
\usepackage{rotating}
\usepackage[normalem]{ulem}
\usepackage{amsmath}
\usepackage{textcomp}
\usepackage{amssymb}
\usepackage{capt-of}
\usepackage{hyperref}
\usepackage{bm}\usepackage[left=1in,right=1in]{geometry}
\date{}
\title{}
\hypersetup{
 pdfauthor={Luis Mochan},
 pdftitle={},
 pdfkeywords={},
 pdfsubject={},
 pdfcreator={Emacs 27.1 (Org mode 9.3)},
 pdflang={English}}
\begin{document}

\noindent{Dear Maurizio Ferrari\\Handling Editor\\Optical Materials}

Please find enclosed a revised version of our manuscript OM-D-21-00419
entitled \emph{Optimization of wide-band quasi-omnidirectional 1-D photonic
structures}. In the following we discuss the reviewer's remarks.
\section{Reviewer \#2:}
\label{sec:orgd1a75c7}
\begin{itemize}
\item \emph{First of all, I would like to commend this work. The article
    overall is very interesting, it contains a good amount of valuable
    information, a detailed description of the methods used, and
    convincing results from both calculations and experiments. The
    body of the paper is for the most part clear and complete, and I
    would recommend it for publication,}
\begin{itemize}
\item We are grateful for the reviewer's opinion.
\end{itemize}
\item \emph{as soon as some minor corrections are performed.}
\item \emph{In particular, there seems to be an issue with the labeling
    of figures and tables: the numbering in the text does not match
    the notations in the figure and table captions.}
\begin{itemize}
\item We are sorry for this confusion which was due to misplaced
\texttt{\textbackslash{}label} statements in the tex source. We have made the
corresponding corrections.
\end{itemize}
\item \emph{Also, in the final paragraph of the results section you
    compare the qODB with the ODB of devices proposed by other papers
    (table 3). While it has been stated before that your results are
    specifically referring to a quasi-omnidirectional PBG, I suggest
    pointing out explicitly one more time whether you are comparing a
    quasi-omindirectional PBG of this paper to the actual
    omnidirectional PBGs from literature, or you are considering just
    the 0-60� qODW for all of the cited works. This is also due to the
    fact that the papers in question present different angular ranges
    in their evaluation of the omnidirectional bandgap, and therefore
    the comparison should take that into account for the sake of
    transparency.}
\begin{itemize}
\item We agree with the reviewer. In the discussion of Table 3 we
now remind the reader that our band is not \emph{omnidirectional} but
\emph{quasi-omnidirectional} and we included the angular range
reported in the corresponding references.
\end{itemize}
\end{itemize}
\section{In summary}
\label{sec:org9c0bedc}
\begin{itemize}
\item We have addressed all of the remarks and suggestions received from
the reviewer.
\item We have further corrected some writing mistakes and made some
changes to our abstract and conclusions to avoid repetitions.
\item We have highlighted in color the revisions to our manuscript.
\item Our manuscript presents different approaches to the design of
optimized wide-band omnidirectional mirrors, we fabricated and
characterized the corresponding structures and obtained mirrors
that compare favorably to others described in the literature.
\item The reviewer commended this work and recommended its publication,
save for the issues which have been addressed in this revision.
\item Thus we believe that the current version is appropriate for
publication in Optical Materials.
\end{itemize}

\noindent Yours truly,

\noindent Victor Castillo-Gallardo
\end{document}